\section{Introducción}

La agricultura moderna enfrenta retos como el aumento de la demanda alimentaria debido al crecimiento poblacional, el cambio climático y el agotamiento de recursos naturales \cite{thayer2020}. Esto resalta la necesidad de optimizar el uso de la tierra para reducir el impacto ambiental y maximizar los beneficios socioeconómicos. Los datos satelitales permiten mapear terrenos agrícolas, gestionar eficientemente recursos críticos como el agua y los fertilizantes, y monitorear la salud de los cultivos, mejorando así la productividad \cite{boryan2011}.

Tradicionalmente, la identificación y el monitoreo de cultivos dependían de observaciones de campo, un proceso intensivo y propenso a errores. Recientemente, las imágenes aéreas capturadas por vehículos no tripulados han facilitado este proceso, proporcionando datos valiosos para la toma de decisiones agrícolas \cite{hu2021}. Las imágenes satelitales se han utilizado ampliamente para clasificar cultivos debido a su riqueza informativa.

Este trabajo propone un modelo de clasificación de cultivos basado en técnicas de Deep Learning, utilizando composiciones de bandas de imágenes satelitales. En primer lugar, se analizará el estado del arte sobre el uso de Deep Learning para la identificación de zonas cultivadas, evaluando enfoques existentes y áreas de mejora. Además, se identificarán características relevantes de las imágenes que contribuyan a mejorar la clasificación. Se implementarán estrategias de transferencia de aprendizaje y ajuste fino (Fine-tuning) para aprovechar modelos preentrenados.

El modelo se aplicará a una región específica de Colombia y se comparará con métodos tradicionales basados en índices de vegetación, destacando las ventajas del enfoque de Deep Learning. La metodología se divide en tres fases: modelado, que incluye la selección de la arquitectura y el preprocesamiento de imágenes de Sentinel-1 y Sentinel-2 \cite{descals2021}; implementación, centrada en el entrenamiento y prueba de modelos preentrenados; y validación, en la cual se evaluarán métricas de desempeño y se analizarán las limitaciones del modelo aplicadas a imágenes representativas del territorio colombiano.
