\section{Conclusiones y Trabajos futuros}

\subsection{Conclusiones}

La investigación desarrollada en esta tesis ha culminado en una serie de hallazgos significativos en el ámbito de la clasificación de cultivos mediante técnicas de Deep Learning, utilizando imágenes satelitales de Sentinel 1 y Sentinel 2. Los resultados obtenidos han demostrado que el modelo propuesto, basado en técnicas avanzadas de aprendizaje profundo, es altamente eficaz en la identificación de plantaciones de palma de aceite, alcanzando una precisión global del 98\%. Esta alta precisión subraya la capacidad del modelo para integrar eficientemente datos de radar y ópticos, lo que resalta el potencial de las imágenes SAR de Sentinel 1 y las ópticas de Sentinel 2 en la clasificación precisa de este tipo de cultivos.

Sin embargo, el estudio también ha identificado ciertas limitaciones en el modelo, especialmente en la detección de áreas recién despejadas y plantaciones jóvenes. Este hallazgo enfatiza la necesidad de incorporar datos adicionales que reflejen la edad y el estado de desarrollo de las plantaciones para una evaluación más precisa y detallada. Además, se observó que la aplicación de técnicas de aumento de datos, como la rotación de imágenes, ha sido fundamental para mejorar la eficacia del modelo. Estas técnicas han probado ser particularmente valiosas en regiones con una cantidad limitada de datos de entrenamiento, demostrando la importancia de un conjunto de datos diversificado y enriquecido en estudios de teledetección.

Al comparar estos resultados con el enfoque más tradicional de aprendizaje automático empleado por \cite{diaz2023}, que utilizó una combinación de algoritmos convencionales y análisis de índices de vegetación, se observa que, aunque estos métodos tradicionales son efectivos, el modelo basado en Deep Learning destaca por su especialización y mayor precisión en la tarea específica de clasificar cultivos de palma de aceite. Este contraste pone en un pequeño relieve la superioridad de las técnicas de Deep Learning en aplicaciones especializadas de teledetección.

\subsection{Trabajo futuro}

Se sugiere entrenar el modelo con más datos, en especial imágenes de plantaciones jóvenes y áreas recién despejadas. Además, explorar el uso de técnicas más avanzadas de procesamiento de imágenes, como redes neuronales generativas, para mejorar la clasificación en condiciones de baja calidad de imagen. Además, se propone extender la investigación a otros tipos de cultivos y condiciones geográficas para evaluar la adaptabilidad del modelo. Un área de interés particular podría ser el análisis de cambios temporales en las plantaciones, utilizando series de tiempo de imágenes satelitales para monitorear el desarrollo y salud de los cultivos. También sería valioso investigar el impacto de diferentes condiciones climáticas y estacionales en la precisión del modelo. Por último, un enfoque en la interpretación y explicabilidad de los modelos de Deep Learning podría proporcionar puntos de vista más profundos sobre cómo y por qué el modelo toma ciertas decisiones, lo cual es crucial para su implementación y confianza en escenarios de la vida real.
