\section{Conclusiones y Trabajos futuros}
Este estudio demuestra la eficacia del uso de técnicas avanzadas de Deep Learning para la clasificación de cultivos de palma de aceite en Colombia, integrando datos satelitales de Sentinel-1 y Sentinel-2. El modelo propuesto, basado en DeepLabV3+ y MobileNet V3, alcanzó una precisión global del 98.37\%, superando significativamente los métodos tradicionales. Estos resultados destacan la capacidad del enfoque para manejar desafíos complejos como la integración de datos radar y ópticos, y su robustez ante condiciones como la cobertura de nubes.

Sin embargo, se identificaron limitaciones, particularmente en la detección de áreas recientemente despejadas y plantaciones jóvenes. Esto sugiere la necesidad de incorporar datos adicionales que reflejen mejor la variabilidad en las etapas de desarrollo de las plantaciones. Además, técnicas como la rotación de imágenes han demostrado ser esenciales para enriquecer los datos de entrenamiento y mejorar la eficacia del modelo, especialmente en regiones con conjuntos de datos limitados.

En comparación con enfoques tradicionales, como los basados en índices de vegetación y algoritmos de aprendizaje automático, el modelo de Deep Learning exhibe una superioridad notable en tareas específicas de clasificación y segmentación de imágenes agrícolas. Este trabajo subraya el potencial de estas herramientas para transformar la agricultura de precisión, no solo en Colombia, sino también en otros contextos globales.

Para futuros estudios, se recomienda ampliar el conjunto de datos, incluir más variedades de cultivos y explorar análisis temporales utilizando series de imágenes satelitales. Asimismo, sería valioso investigar cómo las condiciones climáticas y estacionales afectan la precisión del modelo, además de priorizar la interpretabilidad de los resultados para fomentar su adopción en aplicaciones del mundo real. Estas iniciativas fortalecerán aún más el impacto de las tecnologías de Deep Learning en la sostenibilidad y gestión agrícola.
