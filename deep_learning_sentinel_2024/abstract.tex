\begin{abstract}
Esta tesis aborda el desafío de optimizar la agricultura utilizando técnicas de Deep Learning para la clasificación de cultivos en imágenes satelitales de Sentinel 1 y Sentinel 2, enfocándose en la región de Colombia. Frente a la creciente demanda de alimentos y el cambio climático, se desarrolla un modelo avanzado para mejorar la gestión de recursos agrícolas y la productividad. El estudio comienza con un análisis exhaustivo del estado del arte en la identificación de zonas cultivadas mediante Deep Learning, destacando la necesidad de optimizar el uso de recursos como agua y fertilizantes. Se implementa un modelo basado en la arquitectura DeepLab V3+ con ajustes en MobileNet V3 y ResNet50, para clasificar cultivos de palma de aceite en Colombia. Se emplean estrategias como transferencia de aprendizaje y ajuste fino, utilizando modelos pre entrenados para acelerar el entrenamiento. El diseño metodológico se divide en tres fases: modelado, implementación y validación. Se evalúan métricas de desempeño y se analizan limitaciones, especialmente en la detección de áreas recién despejadas y plantaciones jóvenes. Los resultados muestran una alta precisión del modelo en la clasificación de cultivos de palma de aceite, destacando su eficacia y potencial para aplicaciones prácticas en la gestión agrícola. Se concluye la superioridad del Deep Learning sobre métodos tradicionales, sugiriendo futuras mejoras y ampliación del estudio a otros cultivos y condiciones geográficas.
\end{abstract}

\begin{IEEEkeywords}
Deep Learning, imágenes satelitales, clasificación de cultivos, agricultura, Sentinel.
\end{IEEEkeywords}
