\begin{abstract}
Este estudio presenta un modelo avanzado basado en Deep Learning para la clasificación de plantaciones de palma de aceite en Colombia, utilizando datos satelitales de Sentinel-1 y Sentinel-2. Ante la creciente demanda de alimentos y los desafíos del cambio climático, el enfoque propuesto mejora la gestión agrícola mediante la integración de datos radar y ópticos. El modelo implementado, basado en la arquitectura DeepLabV3+ con MobileNet V3 y ResNet50, emplea estrategias de transferencia de aprendizaje y ajuste fino para optimizar su rendimiento. La metodología comprende tres fases: modelado, implementación y validación, destacándose por una precisión global del 98.37\%. Este trabajo subraya la eficacia del Deep Learning en comparación con métodos tradicionales, superando limitaciones como la cobertura de nubes y resolviendo tareas complejas de segmentación de imágenes. Los resultados demuestran su aplicabilidad práctica en la gestión de cultivos, sugiriendo futuras mejoras para incluir plantaciones jóvenes y explorar otros tipos de cultivos y regiones geográficas.
\end{abstract}

\begin{IEEEkeywords}
Deep Learning, Clasificación de cultivos, Imágenes satelitales, Sentinel-1 y Sentinel-2, Agricultura de precisión.
\end{IEEEkeywords}
