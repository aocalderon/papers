La agricultura moderna se enfrenta a varios retos, entre ellos está la creciente demanda 
de alimentos como consecuencia del aumento de la población mundial, sumado a esto, el
cambio climático junto con el agotamiento de recursos naturales (Thayer et al., 2020). Esto
ha creado la necesitad de optimizar el uso de la tierra para la producción de cultivos con
el fin de disminuir los impactos al medio ambiente y aumentar el impacto socioeconómico.
Se ha demostrado que los datos satelitales permiten realizar mapas agrícolas y globales,
los cuales permiten una gestión más eficiente de recursos críticos como el agua,
fertilizantes y pesticidas. Además, permite el monitoreo de la salud de las plantas y un
aumento de la productividad al optimizar la planificación de la siembra y la cosecha (Boryan
et al., 2011).

Históricamente, la identificación de cultivos y el monitoreo de su evolución se basaron en
observaciones de campo, un proceso laborioso y propenso a errores humanos. Para
reducir este esfuerzo, en tiempos recientes se han utilizado imágenes aéreas de campos
de cultivo, que se adquieren y procesan desde vehículos aéreos no tripulados. Esta
capacidad ha abierto las puertas a la agricultura, permitiendo a los agricultores y a los
responsables de la toma de decisiones acceder a datos valiosos sobre el estado de los
cultivos, la utilización de la tierra y la gestión de recursos (Hu et al., 2021). El uso de
imágenes satelitales para la clasificación de diferentes tipos de cultivos se ha abordado en
estudios anteriores debido a la gran cantidad de información que estas imágenes
contienen.

Esta investigación está centrada en desarrollar un modelo de clasificación avanzado
utilizando técnicas de Deep Learning para identificar y clasificar cultivos a partir de una
composición de bandas de imágenes satelitales. Esta tarea es vital en el contexto agrícola,
especialmente para optimizar el uso de recursos y mejorar las prácticas de gestión de
cultivos. Inicialmente, se llevará a cabo un exhaustivo análisis del estado del arte,
evaluando cómo se han aplicado previamente técnicas de Deep Learning en la
identificación de zonas cultivadas. Esto nos permitirá entender mejor los enfoques actuales
y las posibles áreas de mejora.

El enfoque de este trabajo también incluye identificar las características más relevantes en
las imágenes satelitales que permiten clasificar eficientemente los diferentes tipos de
cultivos. Esta fase es crucial para asegurar la precisión y relevancia del modelo propuesto.
Además, investigaremos y aplicaremos estrategias de transferencia de aprendizaje y
ajuste fino (Fine-tuning) para aprovechar los modelos de Deep Learning pre entrenados.
Esta aproximación tiene como objetivo acelerar y mejorar el proceso de entrenamiento de
nuestro modelo.

Luego, aplicaremos el modelo desarrollado en una zona geográfica específica de
Colombia. Se comprará su rendimiento con un modelo de clasificación basados en índices
de vegetación convencionales, buscando evidenciar las ventajas y mejoras que el enfoque
del Deep Learning puede aportar en el campo de la clasificación de cultivos mediante
imágenes satelitales. Para desarrollar el modelo de clasificación se planteó una
metodología dividida en tres fases. La primera es el modelado donde se destaca la
selección de la arquitectura del modelo y el preprocesamiento de las imágenes satelitales
de Sentinel 1 y Sentinel 2 junto con la composición de estas mismas realizada por Descals
et al. (2021). Luego en la etapa de implementación se centra en entrenamiento del modelo,
así como la selección y prueba de modelo pre entrenados. Por último, la validación del
modelo se realiza específicamente en imágenes de zonas seleccionadas, representativas
del territorio colombiano. En esta última etapa se evalúan las métricas de desempeño del
modelo obtenido junto con el análisis de las limitaciones de este.